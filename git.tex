
\section{Git}

\begin{frame}{Git introduction}

\textit{Git} (2005, Linus Torvalds) is a very powerful VC system.

\vspace{2em}

In this presentation we'll skip description and internal working model, and we'll
focus only on usage.

\end{frame}


\begin{frame}[fragile]{Initialize a git repository}

If we don't have a repository, we need to create it:

\vspace{2em}

\Large \texttt{git init}

\vspace{2em}

\normalsize
\begin{itemize}
  \item Git will create a new (local) repository
  \item A special directory, named \texttt{.git}, is created on the root of the
  project
  \item No file will be added to the repository
\end{itemize}

\end{frame}


\begin{frame}[fragile]{Clone an existing repository}

If a repository exists on a remote endpoint, we can clone it:

\vspace{2em}

\Large \texttt{git clone} \textit{<url>} \textit{[<localdir>]}

\vspace{2em}

\end{frame}


\begin{frame}[fragile]{Adding a file into a repository}

\vspace{2em}

\Large \texttt{git add} \textit{<file>}

\vspace{2em}

\normalsize
Adding a file into a repository means that Git will track the changes on file
contents, attributes and other metadatas.

You need to add a file into the repository both the first time (to inform Git that
you want to track that file) and before every commit where you want to ``save"
the file change (to inform Git that you want to store the change in this commit).

\end{frame}


\begin{frame}[fragile]{Create a commit in Git history}

\vspace{2em}

\Large \texttt{git commit}

\vspace{2em}

Note: if you don't specify the commit message with \texttt{-m} parameter,
Git will open the default editor to write one.

\end{frame}


\begin{frame}[fragile]{Common operations}

Create a new branch \\
\texttt{git branch} \textit{<branchname>}

Create a new branch and switch to it \\
\texttt{git checkout -b} \textit{<branchname>}

Merge the branch \textit{bob} into current branch \\
\texttt{git merge} \textit{bob}

Send commits to (default) remote endpoint \\
\texttt{git push}

Retrieve commits from (default) remote endpoint \\
\texttt{git pull}

\end{frame}
