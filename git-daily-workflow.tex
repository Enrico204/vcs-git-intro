

\section{Git: workstation setup}

\begin{frame}[c,fragile]{Before we begin}

On a new computer/user profile, we need to setup our environment:

\begin{center}
\begin{minipage}{\textwidth}

\begin{listing}[H]
%\scriptsize
\begin{minted}[mathescape,
							 numbersep=5pt,
							 gobble=0,
							 frame=lines,
							 framesep=2mm]{bash}
$ git config --global user.name "John Doe"
$ git config --global user.email john@doe.com
\end{minted}
\end{listing}

\end{minipage}
\end{center}

\end{frame}


\begin{frame}[c,fragile]{Start to develop on an existing Git repository}

If you(/your team) have an existing project on Git, you can start by \textit{cloning}
the repository locally

\begin{center}
\begin{minipage}{\textwidth}

\begin{listing}[H]
%\scriptsize
\begin{minted}[mathescape,
							 numbersep=5pt,
							 gobble=0,
							 frame=lines,
							 framesep=2mm]{bash}
$ git clone https://githosting/gitrepo gitrepodir
$ cd gitrepodir/
\end{minted}
\end{listing}

\end{minipage}
\end{center}

\end{frame}


\begin{frame}[c,fragile]{Import an existing project into a repository}

If you have an existing project outside Git, you can import it with:

\begin{center}
\begin{minipage}{\textwidth}

\begin{listing}[H]
%\scriptsize
\begin{minted}[mathescape,
							 numbersep=5pt,
							 gobble=0,
							 frame=lines,
							 framesep=2mm]{bash}
$ cd existingproject/
$ git init
$ git add .
$ git commit
$ git remote add origin https://githosting/gitrepo
$ git push -u origin master
\end{minted}
\end{listing}

\end{minipage}
\end{center}

\end{frame}




\section{Git: daily workflow}

\begin{frame}[c,fragile]{Pulling remote changes}

At the beginning of the day, or when someone push changes to server, you need to
update your local copy:

\begin{center}
\begin{minipage}{\textwidth}

\begin{listing}[H]
%\scriptsize
\begin{minted}[mathescape,
							 numbersep=5pt,
							 gobble=0,
							 frame=lines,
							 framesep=2mm]{bash}
$ cd existingproject/
$ git pull
\end{minted}
\end{listing}

\end{minipage}
\end{center}

\end{frame}


\begin{frame}[c,fragile]{Creating branch}

When needed, you can create a new branch, starting from the current commit:

\begin{center}
\begin{minipage}{\textwidth}

\begin{listing}[H]
%\scriptsize
\begin{minted}[mathescape,
							 numbersep=5pt,
							 gobble=0,
							 frame=lines,
							 framesep=2mm]{bash}
$ cd existingproject/
$ git branch new-feature-1
$ git checkout new-feature-1
$
\end{minted}
\end{listing}

\end{minipage}
\end{center}

\end{frame}


\begin{frame}[c,fragile]{Merging branch}

Let's suppose that we're on \texttt{master} branch and we want to merge \texttt{new-feature-1}

\begin{center}
\begin{minipage}{\textwidth}

\begin{listing}[H]
%\scriptsize
\begin{minted}[mathescape,
							 numbersep=5pt,
							 gobble=0,
							 frame=lines,
							 framesep=2mm]{bash}
$ cd existingproject/
$ git merge new-feature-1
$ git push
$
\end{minted}
\end{listing}

\end{minipage}
\end{center}

\end{frame}


\begin{frame}[c,fragile]{Commit and push after work}

At the end of the day, or when needed, we can create a commit by issuing:

\begin{center}
\begin{minipage}{\textwidth}

\begin{listing}[H]
%\scriptsize
\begin{minted}[mathescape,
							 numbersep=5pt,
							 gobble=0,
							 frame=lines,
							 framesep=2mm]{bash}
$ cd existingproject/
$ git add modifiedfile1 addedfile2 removedfile3
$ git commit
$ git push
\end{minted}
\end{listing}

\end{minipage}
\end{center}

\end{frame}
